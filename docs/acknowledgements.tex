\custompagebreak

\begingroup
\renewcommand{\thepage}{}

\bookmarksetupnext{italic}
\pdfbookmark[0]{Acknowledgements}{acknowledgements}

\vspace*{-2.5em}

\epigraph{\epigraphfont\large
  The Road goes ever on and on,\\
  Down from the door where it began.\\
  Now far ahead the Road has gone,\\
  And I must follow, if I can,\\
  Pursuing it with eager feet,\\
  Until it joins some larger way\\
  Where many paths and errands meet.\\
  And whither then? I cannot say.
}{---~J.~R.~R. Tolkien, \textit{The Fellowship of the Ring}}

\vspace{-1.5em}

\section*{Acknowledgements}
The road of this thesis research has often joined with the paths of others, and it has been a privilege to travel it with so many generous companions. I would not even have found the way without their guidance and support, and I feel fortunate for the lessons, inspiration, and friendship they provided throughout.

Above all, I would like to thank my advisor, \mbox{Prof. Dr.} Miriam Fritsch, without whom I would not have embarked on this journey in the first place. I am especially thankful for the regular discussions she organised, for the insightful feedback she provided on the different directions I pursued, and for the freedom she gave me to follow new ideas in developing the analysis framework. I am likewise grateful for the many opportunities she gave me to participate in meetings within the fields of hadron physics and computing.

Several of the most important collaborations of this thesis began at such meetings. Chief among them has been my collaboration with \mbox{Prof. Dr.} Mikhail Mikhasenko. Since the polarimetry studies, he has guided me in correctly implementing and verifying spin-aligned amplitude models. This work became an important foundation of the results presented in this thesis and would not have been possible without his expertise and support.

In the initial stages of this work, I benefited greatly from the advice and encouragement of \mbox{Dr. Stefan} Pflüger, who introduced me to the world of amplitude analysis and the existing software landscape. Our daily programming sessions, even when remote during the pandemic, shaped my understanding of good software design as we reworked the framework into its current symbolic and array-oriented form. I would also like to express my gratitude to \mbox{Prof. Dr.} Wolfgang Gradl, whose feedback and perspective during our group meetings were important in shaping the features of the analysis framework.

Alongside these collaborations, the day-to-day life at the institutes mattered just as much. I~am grateful to my office mates and colleagues, first in Mainz and later in Bochum, for the many discussions during lunch and coffee breaks. Over the years, our office became a place where serious work was balanced with the humour we shared each day, and in the later stages of my dissertation, I especially valued how they motivated me to stay focused on the writing. I also owe much to my friends~-- those I met in Bochum, who made the city feel like home, and those in the Netherlands, whose support from afar has accompanied me throughout.

Finally, I would like to thank my mother and my brother; the regular trips back home were not only a welcome way to recharge my batteries, but also helped me stay grounded outside my research. My father inspired me to pursue a career in physics and nurtured my interest in computing, and although he sadly did not live to see this part of the journey, his backing and encouragement continue to accompany me.

\endgroup
